% LaTeX resume using res.cls
\documentclass[line,margin,letterpaper]{res} 

\addtolength{\oddsidemargin}{-.5in}
\addtolength{\evensidemargin}{-.5in}
\addtolength{\textwidth}{1.0in}
\addtolength{\topmargin}{-.7in}
\addtolength{\textheight}{1.5in}

\begin{document}

\name{\Large Alfredo E. Miranda}
\address{alfredo@miranda.io - http://miranda.io}

\begin{resume}
 
\section{EDUCACI\'ON}
  {\bf The University of North Carolina at Chapel Hill} 
  \hfill Chapel Hill, NC \\
  \emph{M.S., Computer Science, GPA: 3.53/4.00} \hfill Agosto 2013

  {\bf The University of Texas at Austin} \hfill Austin, TX \\
  \emph{B.S., Electrical Engineering, GPA: 3.64/4.00} \hfill  Agosto 2011 \\
  \emph{B.S., Mathematics}
 
\section{EXPERIENCIA} 
  {\bf NextGen Healthcare} \hfill Austin, TX \\
  \emph{Ingeniero de Software} \hfill Marzo 2014-Presente
  \begin{itemize} \itemsep -2pt
    \item Usando HTML5, CSS3, JavaScript, y AngularJS para desarrollar 
    aplicaci\'on web
  \end{itemize}
  {\bf Departamento de Ciencias de la Computaci\'on de UNC} \hfill 
    Chapel Hill, NC \\
  \emph{Asistente de Investigaci\'on} \hfill Agosto 2011-Julio 2013
  \begin{itemize} \itemsep -2pt
    \item Lider\'e la iniciativa de investigaci\'on para modelar TCP Rapid y 
    obtener control de congestion eficiente
    \item Cre\'e modelo del throughput de TCP Rapid en presencia de una simple 
    fuente de on-off cross traffic para identificar valores de parametros que 
    optimizen el rendimiento
    \item Valid\'e modelos usando simulaciones de ns-2
    \item Present\'e resultados de investigaciones a grupos de investigadores
    y facilite el panel de discusi\'on
  \end{itemize}
  {\bf Smarts and Stamina} \hfill Chapel Hill, NC \\
  \emph{Desarrollador de Software} \hfill Enero 2013-Mayo 2013
  \begin{itemize} \itemsep -2pt
    \item Trabaj\'e en un ambiente colaborativo para desarrollar y probar 
    aplicaci\'on mobil de fitness personal
    \item Program\'e en HTML, CSS, y Javascript, y use el framework PhoneGap 
    para hacer funcionar la aplicaci\'on en multiples plataformas m\'obiles
  \end{itemize}
  {\bf Departamento de Ciencias de la Computaci\'on de UNC} 
    \hfill Chapel Hill, NC \\
  \emph{Profesor Auxiliar} \hfill Agosto 2012-Mayo 2013
  \begin{itemize} \itemsep -2pt
    \item Asist\'i a la profesora Jasleen Kaur Ph.D. en corregir ex\'amenes 
    parciales y finales para clase de Servicios de Internet y Protocolos
    \item Desarroll\'e programa en Python para automatizar la correcci\'on de 
    tareas de programaci\'on y reducir el tiempo de correcci\'on 
    considerablemente
    \item Encargado de la secci\'on de discusi\'on de la clase
  \end{itemize}
  {\bf Departamento de Ingenier\'ia Mec\'anica de UT Austin} 
    \hfill Austin, TX \\
  \emph{Asistente de Investigaci\'on} \hfill Septiembre 2010-Mayo 2011
  \begin{itemize} \itemsep -2pt
    \item Escrib\'i documentaci\'on y tutorial para usar BriPy, un modulo de 
    Python para el modelo del ciclo del combustible nuclear Bright
    \item Desarroll\'e aplicaci\'on de flujo de trabajo para que los usuarios 
    manipulen visualmente componentes del modelo del ciclo de combustible 
    nuclear Bright
  \end{itemize}
  {\bf The University of Texas at Austin} \hfill Austin, TX \\
  \emph{Tutor de Matem\'aticas} \hfill Enero 2008-Mayo 2011
  \begin{itemize} \itemsep -2pt
    \item Ense\~n\'e a estudiantes en entornos individuales y grupales
    \item Desarroll\'e ejemplos de ex\'amenes para ayudar a los estudiantes a 
    prepararse para sus ex\'amenes parciales y finales
  \end{itemize}
 
\section{PUBLICACI\'ON}
  A. Miranda, C-W. Kan, D. Cote, K. Sokolov, M. K. Markey, ``Visualization 
  Tools for Pol-MC to Simulate Polarized Light-Tissue Interaction'', 
  Biomedical Engineering Society Annual Meeting (2010).

\section{HABILIDADES}
  \emph{Lenguajes de Programaci\'on:} JavaScript, Python, Java, y C \\
  \emph{Lenguajes Markup:} HTML5 y \LaTeX \\
  \emph{Tecnolog\'ias Web:} CSS3, jQuery, y AngularJS \\
  \emph{Sistema de Control de Versiones:} Git \\
  Dominio del Ingl\'es

\end{resume}
\end{document}
