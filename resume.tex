% LaTeX resume using res.cls
\documentclass[line,margin,letterpaper]{res} 

\addtolength{\oddsidemargin}{-.5in}
\addtolength{\evensidemargin}{-.5in}
\addtolength{\textwidth}{1.0in}
\addtolength{\topmargin}{-.7in}
\addtolength{\textheight}{1.5in}

\begin{document}

\name{\Large Alfredo E. Miranda}
\address{(512) 739-9634 - alfredo@miranda.io - http://miranda.io}

\begin{resume}
 
\section{EDUCATION} 
  {\bf The University of North Carolina at Chapel Hill} 
  \hfill Chapel Hill, NC \\
  \emph{M.S., Computer Science, GPA: 3.53/4.00} \hfill August 2013

  {\bf The University of Texas at Austin} \hfill Austin, TX \\
  \emph{B.S., Electrical Engineering, GPA: 3.64/4.00} \hfill  August 2011 \\
  \emph{B.S., Mathematics}
 
\section{EXPERIENCE}
  {\bf Rapid7} \hfill Austin, TX \\
  \emph{Front End Engineer II} \hfill November 2015-Present
  \begin{itemize} \itemsep -2pt
    \item Developing new features for application currently in production
    \item Using Backbone.js for the front end
  \end{itemize}
  {\bf Mirth Corporation} \hfill Austin, TX \\
  \emph{Software Engineer} \hfill April 2015-October 2015
  \begin{itemize} \itemsep -2pt
    \item Worked with HTML5, CSS3, JavaScript, and AngularJS to develop
    single-page web application
    \item Developed bower component containing a set of angular modules,
    directives, services, and CSS styles that is being used for the rapid
    development of a suite of products
    \item Worked remotely for company located in California
  \end{itemize}
  {\bf NextGen Healthcare} \hfill Austin, TX \\
  \emph{Software Engineer} \hfill March 2014-April 2015
  \begin{itemize} \itemsep -2pt
    \item Worked with HTML5, CSS3, JavaScript, AngularJS, NodeJS, Express, and
    MongoDB to develop single-page web application
    \item Used Git in a feature branch workflow to collaborate with other
    engineers, making sure only quality code gets merged into the master branch
    \item Worked in a continuous integration environment using Jenkins to
    deploy the application several times per day
  \end{itemize}
  {\bf UNC Computer Science Department} \hfill Chapel Hill, NC \\
  \emph{Graduate Research Assistant} \hfill August 2011-July 2013
  \begin{itemize} \itemsep -2pt
    \item Led the TCP Rapid modeling research initiative to achieve 
    ultra-high speed congestion control
    \item Created TCP Rapid throughput model in the presence of a simple 
    periodic on-off cross-traffic source to identify parameter values that 
    optimize performance
    \item Validated models by running ns-2 simulations
%    \item Presented research results to collegiate research groups and 
%    facilitated panel discussion
  \end{itemize}
  {\bf Smarts and Stamina} \hfill Chapel Hill, NC \\
  \emph{Software Developer} \hfill January 2013-May 2013
  \begin{itemize} \itemsep -2pt
    \item Worked in a collaborative environment to develop and quality 
    assurance test personal fitness mobile application
    \item Wrote HTML, CSS, and Javascript, and used PhoneGap framework to make 
    the application work in different mobile platforms
  \end{itemize}
%  {\bf UNC Computer Science Department} 
%  \hfill Chapel Hill, NC \\
%  \emph{Teacher Assistant} \hfill August 2012-May 2013
%  \begin{itemize} \itemsep -2pt
%    \item Assisted professor in grading mid-terms and finals for the Internet 
%    Services and Protocols class
%    \item Developed Python script to automate grading of programming 
%    assignments to reduce grading time considerably
%    \item Led collaborative open discussion section of the class
%  \end{itemize}
  {\bf UT Austin Mechanical Engineering Department} \hfill Austin, TX \\
  \emph{Undergraduate Researcher} \hfill September 2010-May 2011
  \begin{itemize} \itemsep -2pt
    \item Built workflow application for users to visually drag, drop, and 
    connect components of the Bright nuclear fuel cycle model
    \item Wrote documentation and tutorial for using BriPy, a set of Python
    bindings for the Bright nuclear fuel cycle model
  \end{itemize}
%  {\bf UT Austin Biomedical Engineering Department} \hfill Austin, TX \\
%  \emph{Undergraduate Researcher} \hfill May 2009-August 2010
%  \begin{itemize} \itemsep -2pt
%    \item Utilized simulation written in C++ to study the patterns of light 
%    scattering when photons interact with human tissue
%    \item Wrote a script to run the simulation several times and make plots in 
%    MATLAB with the output of the simulation
%  \end{itemize}
%  {\bf The University of Texas at Austin} \hfill Austin, TX \\
%  \emph{Math Tutor} \hfill January 2008-May 2011
%  \begin{itemize} \itemsep -2pt
%    \item Tutored students in one-on-one and group settings
%    \item Worked through sample tests to help students better prepare for 
%    mid-terms and finals
%  \end{itemize}
 
\section{PUBLICATIONS} 
  A. Miranda, C-W. Kan, D. Cote, K. Sokolov, M. K. Markey, ``Visualization 
  Tools for Pol-MC to Simulate Polarized Light-Tissue Interaction'', 
  Biomedical Engineering Society Annual Meeting (2010).

\section{SKILLS}
  \emph{Programming Languages:} JavaScript, Java, and Python\\
  \emph{Markup Languages:} HTML5 and \LaTeX \\
  \emph{Web Technologies:} AngularJS, NodeJS, Express, Backbone.js, CSS3, Sass,
  and jQuery \\
  \emph{Databases:} MongoDB \\
  \emph{Version Control System:} Git \\
%  \emph{Software Development Environments:} Eclipse and Xcode \\
%  \emph{Operating Systems:} UNIX, OS X, and Windows \\
  Fluent in Spanish

\end{resume}
\end{document}
