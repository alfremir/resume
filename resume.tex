% LaTeX resume using res.cls
\documentclass[line,margin]{res} 

\addtolength{\oddsidemargin}{-.5in}
\addtolength{\evensidemargin}{-.5in}
\addtolength{\textwidth}{1.0in}
\addtolength{\topmargin}{-.7in}
\addtolength{\textheight}{1.5in}

\begin{document}

\name{\Large Alfredo E. Miranda}
\address{2036 Amur Drive, Austin, TX 78745 - 512.739.9634 - alfremir@gmail.com}

\begin{resume}
 
\section{EDUCATION} 
  {\bf The University of North Carolina at Chapel Hill} 
  \hfill Chapel Hill, NC \\
  \emph{M.S., Computer Science, GPA: 3.53/4.00} \hfill August 2013

  {\bf The University of Texas at Austin} \hfill Austin, TX \\
  \emph{B.S., Electrical Engineering, GPA: 3.64/4.00} \hfill  August 2011 \\
  \emph{B.S., Mathematics}
 
\section{EXPERIENCE} 
  {\bf UNC Computer Science Department} \hfill Chapel Hill, NC \\
  \emph{Graduate Research Assistant} \hfill August 2011-July 2013
  \begin{itemize} \itemsep -2pt
    \item Led the TCP Rapid modeling research initiative to achieve 
    ultra-high speed congestion control
    \item Created TCP Rapid throughput model in the presence of a simple 
    periodic on-off cross-traffic source to identify parameter values that 
    optimize performance
    \item Validated models by running ns-2 simulations
    \item Presented research results to collegiate research groups and 
    facilitated panel discussion
  \end{itemize}
  {\bf Smarts and Stamina} \hfill Chapel Hill, NC \\
  \emph{Software Developer} \hfill January 2013-May 2013
  \begin{itemize} \itemsep -2pt
    \item Worked in a collaborative environment to develop and quality 
    assurance test personal fitness mobile application
    \item Wrote HTML, CSS, and Javascript, and used PhoneGap framework to make 
    the application work in different mobile platforms
  \end{itemize}
  {\bf UNC Computer Science Department} 
  \hfill Chapel Hill, NC \\
  \emph{Teacher Assistant} \hfill August 2012-May 2013
  \begin{itemize} \itemsep -2pt
    \item Assisted professor in grading mid-terms and finals for the Internet 
    Services and Protocols class
    \item Developed Python script to automate grading of programming 
    assignments to reduce grading time considerably
    \item Led collaborative open discussion section of the class
  \end{itemize}
  {\bf UT Austin Mechanical Engineering Department} \hfill Austin, TX \\
  \emph{Undergraduate Researcher} \hfill September 2010-May 2011
  \begin{itemize} \itemsep -2pt
    \item Wrote documentation and tutorial for using BriPy, a set of Python
    bindings for the Bright nuclear fuel cycle model
    \item Built workflow application for users to visually drag, drop, and 
    connect components of the Bright nuclear fuel cycle model
  \end{itemize}
  {\bf UT Austin Biomedical Engineering Department} \hfill Austin, TX \\
  \emph{Undergraduate Researcher} \hfill May 2009-August 2010
  \begin{itemize} \itemsep -2pt
    \item Utilized simulation written in C++ to study the patterns of light 
    scattering when photons interact with human tissue
    \item Wrote a script to run the simulation several times and make plots in 
    MATLAB with the output of the simulation
  \end{itemize}
  {\bf The University of Texas at Austin} \hfill Austin, TX \\
  \emph{Math Tutor} \hfill January 2008-May 2011
  \begin{itemize} \itemsep -2pt
    \item Tutored students in one-on-one and group settings
    \item Worked through sample tests to help students better prepare for 
    mid-terms and finals
  \end{itemize}
 
\section{PUBLICATIONS} 
  A. Miranda, C-W. Kan, D. Cote, K. Sokolov, M. K. Markey, ``Visualization 
  Tools for Pol-MC to Simulate Polarized Light-Tissue Interaction'', 
  Biomedical Engineering Society Annual Meeting (2010).

\section{SKILLS}
  \emph{Programming Languages:} Java, C, Ruby, Python, MATLAB, and Freescale 
  6812 \\
  \emph{Markup Languages:} HTML and \LaTeX \\
  \emph{Version Control System:} Git \\
  \emph{Software Development Environments:} Eclipse, Xcode, and Visual C++ \\
  \emph{Operating Systems:} UNIX, OS X, and Windows \\
  Fluent in Spanish

\end{resume}
\end{document}
